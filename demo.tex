\documentclass[print]{atomathematyk}

\usepackage{longtable}

% ────────────────────────────────────────────────────────────────────────────────
% (METADATA)

\title{Commandes du package \texttt{maquereaux.sty}}
\author{Léo Guillon}
\date{}

\begin{document}

\maketitle

\section{Logique de conception des commandes}

De manière générale, on essaie de suivre les conventions mathématiques : une commande pour un ensemble est en majuscule, tandis qu’une fonction ou un opérateur sera en minuscule.

Par ailleurs, pour être le plus consistant possible avec les autres commandes déjà existantes en \LaTeX, les commandes doivent être nommées en anglais, dans la mesure du possible et de la convenance.

\section{Listes des commandes}
\centering
\begin{longtable}{lcl}
  \toprule
  Commande & Affichage & Signification\\
  \midrule
  \multicolumn{3}{l}{\strong{Mise en forme mathématique}}\\
  \texttt{mathbi\{X\}} & \(\mathbi{X}\) & lettre en gras et italique\\
  \texttt{mathset\{N\}} & \(\mathset{N}\) & notation des ensembles usuels\\
  \midrule
  \multicolumn{3}{l}{\strong{Généralités}}\\
  \multicolumn{3}{l}{\emph{Constantes mathématiques}}\\
  \texttt{e} & \(\e\) & constante exponentielle\\
  \texttt{i} & \(\i\) & nombre \(\i\)\\
  \texttt{j} & \(\j\) & nombre \(\j:=\e^{\i \TAU/3}=-\frac{1}{2}+\i \frac{\sqrt{3}}{2}\)\\
  \texttt{PI} & \(\PI\) & constante du cercle\\
  \texttt{TAU} & \(\TAU\) & \emph{vraie} constante du cercle\\
  \texttt{gold} & \(\gold\) & nombre d’or\\
  \multicolumn{3}{l}{\emph{Opérateurs génériques}}\\
  \texttt{equalsdef} & \(\equalsdef\) & égal par définition à…\\
  \texttt{kro\{i\}\{j\}} & \(\kro{i}{j}\) & symbole de Kronecker\\
  \texttt{ind} & \(\ind\) & fonction indicatrice\\
  \texttt{inv\{x\}} & \(\inv{x}\) & inverse de \(x\)\\
  \midrule
  \multicolumn{3}{l}{\strong{Théorie des ensembles}}\\
  \texttt{Set\{x,y,…\}} & \(\Set{x,y,…}\) & ensemble quelconque\\
  \texttt{SetP\{x\backslash in E\}\{P(x)\}} & \(\SetP{x \in E}{P(x)}\) & ensemble décrit par une propriété \(P\)\\
  \texttt{SetA\{f(x)\}\{x \backslash in E\}} & \(\SetA{f(x)}{x \in E}\) & ensemble décrit par une fonction \(f\)\\
  \multicolumn{3}{l}{\emph{Opérateurs ensemblistes}}\\
  \texttt{card\{E\}} & \(\card{E}\) & cardinal de l’ensemble \(E\)\\
  \texttt{parts\{E\}} & \(\parts{E}\) & ensemble des parties de l’ensemble \(E\)\\
  \texttt{comp\{E\}} & \(\comp{E}\) & complémentaire de l’ensemble \(E\)\\
  \texttt{inter} & \(\inter\) & intersection\\
  \texttt{union} & \(\union\) & union\\
  \texttt{nor} & \(\nor\) & différence symétrique entre deux ensembles\\
  \texttt{tribeng\{C\}} & \(\tribeng{\mathcal{C}}\) & tribu engendrée par une classe \(\mathcal{C}\)\\
  \texttt{borel} & \(\borel\) & tribu borélienne\\
  \multicolumn{3}{l}{\emph{Applications}}\\
  \texttt{longto} & \(\longto\) & longue flèche \\
  \texttt{function\{f\}\{A\}\{B\}\{x\}\{f(x)\}} & \(\function{f}{A}{B}{x}{f(x)}\) & définition de fonction \\
  \texttt{inj} & \(\inj\) & injection\\
  \texttt{surj} & \(\surj\) & surjection\\
  % \texttt{bij} & \(\bij\) & bijection\\
  \multicolumn{3}{l}{\emph{Ensembles usuels}}\\
  \texttt{N} & \(\N\) & entiers naturels\\
  \texttt{Z} & \(\Z\) & entiers relatifs\\
  \texttt{Q} & \(\Q\) & nombres rationnels\\
  \texttt{R} & \(\R\) & nombres réels\\
  \texttt{C} & \(\C\) & nombres complexes\\
  \texttt{H} & \(\H\) & quaternions\\
  \texttt{K} & \(\K\) & corps usuel de nombres, \(\R\) ou \(\C\)\\
  \texttt{F} & \(\F\) & corps fini\\
  \midrule
  \multicolumn{3}{l}{\strong{Algèbre}}\\
  \multicolumn{3}{l}{\emph{Algèbre générale}}\\
  \texttt{iso} & \(\iso\) & relation d’isomorphisme\\
  \texttt{subgroup} & \(\subgroup\) & relation de sous-groupe\\
  \texttt{normal} & \(\normal\) & relation de sous-groupe normal\\
  \texttt{centre\{G\}} & \(\centre{G}\) & centre du groupe \(G\).\\
  \texttt{eng\{A\}} & \(\eng{A}\) & sous-groupe engendré par \(A\)\\
  \texttt{semiprod} & \(\semiprod\) & produit semi-direct de groupes\\
  \texttt{indice\{H\}\{G\}} & \(\indice{H}{G}\) & indice de \(H\) dans \(G\)\\
  \texttt{action} & \(\action\) & action de groupe\\
  \texttt{orbite\{x\}} & \(\orbite{x}\) & orbite de \(x\)\\
  \texttt{stab\{x\}} & \(\stab{x}\) & stabilisateur de \(x\)\\
  \texttt{fix\{g\}} & \(\fix{g}\) & fixateur de \(g\)\\
  \texttt{groupdual\{G\}} & \(\groupdual{G}\) & groupe dual de \(G\)\\
  \texttt{ideng\{a\}} & \(\ideng{a}\) & idéal engendré par \(a\)\\
  \texttt{Hom} & \(\Hom\) & (homo)morphismes de groupes \\
  \texttt{Iso} & \(\Iso\) & isomorphismes de groupes \\
  \texttt{End} & \(\End\) & endomorphismes de groupes \\
  \texttt{Aut} & \(\Aut\) & automorphismes de groupes \\
  \texttt{Sym} & \(\Sym\) & groupe symétrique\\
  \texttt{Alt} & \(\Alt\) & groupe alterné\\
  \texttt{sign} & \(\sign\) & signature\\
  \texttt{Diedral} & \(\Diedral\) & groupe diédral\\
  \multicolumn{3}{l}{\emph{Algèbre linéaire}}\\
  \texttt{Lin} & \(\Lin\) & ensemble d’applications linéaires\\
  \texttt{Quad} & \(\Quad\) & ensemble de formes quadratiques\\
  \texttt{dual\{E\}} & \(\dual{E}\) & espace dual de l’espace vectoriel \(E\)\\
  \texttt{GL} & \(\GL\) & groupe linéaire\\
  \texttt{SL} & \(\SL\) & groupe spécial linéaire\\
  \texttt{Orth} & \(\Orth\) & groupe orthogonal\\
  \texttt{SO} & \(\SO\) & groupe spécial orthogonal\\
  \texttt{PGL} & \(\PGL\) & groupe linéaire projectif\\
  \texttt{PSL} & \(\PSL\) & groupe spécial projectif\\
  \texttt{M} & \(\M\) & ensemble de matrices\\
  \texttt{S} & \(\S\) & ensemble de matrices symétriques\\
  % \texttt{T\{A\}} & \(\T{A}\) & transposée de la matrice \(A\)\\
  \texttt{adj\{A\}} & \(\adj{A}\) & adjoint de la matrice \(A\)\\
  \texttt{tr} & \(\tr\) & trace\\
  \texttt{ker} & \(\ker\) & noyau\\
  \texttt{im} & \(\im\) & image\\
  \texttt{rg} & \(\rg\) & rang\\
  \texttt{orth\{A\}} & \(\orth{A}\) & orthogonal de \(A\)\\
  \texttt{oorth\{B\}} & \(\oorth{B}\) & orthogonal de \(B\) (dual)\\
  \texttt{codim} & \(\codim\) & codimension\\
  \texttt{com} & \(\com\) & comatrice\\
  \texttt{spectrum} & \(\spectrum\) & spectre\\
  \texttt{spradius} & \(\spradius\) & rayon spectral\\
  \midrule
  \multicolumn{3}{l}{\strong{Analyse}}\\
  \multicolumn{3}{l}{\emph{Topologie}}\\
  \texttt{interior\{A\}} & \(\interior{A}\) & intérieur de \(A\)\\
  \texttt{adh\{A\}} & \(\adh{A}\) & adhérence de \(A\)\\
  \texttt{front\{A\}} & \(\front{A}\) & frontière de \(A\)\\
  \texttt{abs\{x\}} & \(\abs{x}\) & valeur absolue (ou module) de \(x\)\\
  \texttt{norme\{x\}} & \(\norme{x}\) & norme de \(x\)\\
  \texttt{triple\{u\}} & \(\triple{u}\) & norme triple de \(u\)\\
  \texttt{Class} & \(\Class\) & ensemble de fonctions continues\\
  \texttt{Lip} & \(\Lip\) & ensemble de fonctions lipschitziennes\\
  \multicolumn{3}{l}{\emph{Analyse complexe}}\\
  \texttt{conj\{z\}} & \(\conj{z}\) & conjugué du complexe \(z\)\\
  \texttt{Arg} & \(\Arg\) & argument principal\\
  \texttt{Log} & \(\Log\) & logarithme principal\\
  \texttt{Circunit} & \(\Circunit\) & cercle unité du plan complexe\\
  \texttt{Diskunit} & \(\Diskunit\) & disque unité du plan complexe\\
  \texttt{Anal} & \(\Anal\) & ensemble de fonctions analytiques\\
  \texttt{Holo} & \(\Holo\) & ensemble de fonctions holomorphes\\
  \texttt{Mero} & \(\Mero\) & ensemble de fonctions méromorphes\\
  \multicolumn{3}{l}{\emph{Analyse fonctionnelle}}\\
  \texttt{L} & \(\L\) & espace \(\L^p\)\\
  \texttt{l} & \(\l\) & espace \(\l^p\)\\
  \multicolumn{3}{l}{\emph{Calcul différentiel et intégral}}\\
  \texttt{d} & \(\d\) & opérateur différentiel élémentaire\\
  \texttt{diff\{f\}\{a\}} & \(\diff{f}{a}\) & différentielle de \(f\) en \(a\)\\
  \texttt{grad} & \(\grad\) & gradient\\
  % \texttt{div} & \(\div\) & divergence\\
  \texttt{rot} & \(\rot\) & rotationnel\\
  \texttt{lap} & \(\lap\) & laplacien\\
  \texttt{jac} & \(\jac\) & matrice jacobienne\\
  \texttt{detjac\{f\}\{a\}} & \(\detjac{f}{a}\) & déterminant jacobien de \(f\) en \(a\)\\
  \texttt{hess\{f\}} & \(\hess{f}\) & matrice hessienne de \(f\) en \(a\)\\
  \midrule
  \multicolumn{3}{l}{\strong{Probabilités}}\\
  \multicolumn{3}{l}{\emph{Opérateurs usuels}}\\
  \texttt{Prob\{A\}} & \(\Prob{A}\) & probabilité d’un évènement \(A\)\\
  \texttt{tops} & \(\tops\) & convergence presque sûre\\
  \texttt{toprob} & \(\toprob\) & convergence en probabilité\\
  \texttt{tolaw} & \(\tolaw\) & convergence en loi\\
  \texttt{toL\{p\}} & \(\toL{p}\) & convergence \(\L^p\)\\
  \texttt{Esp\{X\}} & \(\Esp{X}\) & espérance d’une variable aléatoire \(X\)\\
  \texttt{Var\{X\}} & \(\Var{X}\) & variance d’une variable aléatoire \(X\)\\
  \texttt{sd\{X\}} & \(\sd{X}\) & écart-type d’une variable aléatoire \(X\)\\
  \multicolumn{3}{l}{\emph{Lois discrètes usuelles}}\\
  \texttt{Bernoulli\{p\}} & \(\Bernoulli{p}\) & loi de Bernoulli de paramètre \(p\)\\
  \texttt{Binom\{n\}\{p\}} & \(\Binom{n}{p}\) & loi binomiale de paramètres \((n,p)\)\\
  \texttt{Poisson\{\backslash lambda\}} & \(\Poisson{\lambda}\) & loi de Poisson de paramètre \(\lambda\)\\
  \texttt{Geom\{p\}} & \(\Geom{p}\) & loi géométrique de paramètre \(p\)\\
  \texttt{Hyper\{N\}\{n\}\{k\}} & \(\Hyper{N}{n}{k}\) & loi hypergéométrique de paramètres \((N,n,k)\)\\
  \multicolumn{3}{l}{\emph{Lois continues usuelles}}\\
  \texttt{Exp\{\backslash lambda\}} & \(\Exp{\lambda}\) & loi exponentielle de paramètre \(\lambda\)\\
  \texttt{Normale\{\backslash mu\}\{\backslash sigma 2\}} & \(\Normale{\mu}{\sigma^2}\) & loi normale de paramètres \((\mu, \sigma)\)\\
  \texttt{chid\{n\}} & \(\chi2{n}\) & loi du \(\chi^2\) à \(n\) degrés de liberté\\
  \midrule
  \multicolumn{3}{l}{\strong{Arithmétique}}\\
  \texttt{Zmod\{n\}} &  \(\Zmod{n}\) & classe d’équivalence modulo \(n\) \\
  \texttt{Primes} & \(\Primes\) & ensemble des nombres premiers \\
  \texttt{divides} & \(\divides\) & relation de divisibilité \\
  \texttt{congru\{a\}\{b\}\{n\}} & \(\congru{a}{b}{n}\) & \(a\) congru à \(b\) modulo \(n\)\\
  \texttt{pgcd\{a\}\{b\}} & \(\pgcd{a}{b}\) & PGCD \(a\) et \(b\)\\
  \texttt{ppcm\{a\}\{b\}} & \(\ppcm{a}{b}\) & PPCM \(a\) et \(b\)\\
  \texttt{indeuler} & \(\indeuler\) & fonction indicatrice d’Euler\\
  \texttt{legendre\{a\}\{q\}} & \(\legendre{a}{q}\) & symbole de legendre entre \(a\) et \(q\)\\
  \bottomrule
\end{longtable}

\end{document}
