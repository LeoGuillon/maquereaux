\documentclass{article}

\usepackage{maquereaux}

\usepackage{booktabs}

% \usepackage{sectsty}
% \allsectionsfont{\sffamily}

\usepackage{fontspec}
\setmainfont{IBM Plex Serif}
\setsansfont{IBM Plex Sans}
\setmonofont{IBM Plex Mono}

\usepackage{unicode-math}
\setmathfont{IBM Plex Math}

% ────────────────────────────────────────────────────────────────────────────────
% (METADATA)

\title{Commandes du package \texttt{maquereaux.sty}}
\author{Léo Guillon}
\date{}

\begin{document}

\maketitle

\begin{tabular}{lcl}
  \toprule
  Commande & Affichage & Signification\\
  \midrule
  \multicolumn{3}{c}{\strong{Généralités}}\\
  \midrule
  \texttt{e} & $\e$ & constante exponentielle\\
  \texttt{i} & $\i$ & nombre $\i$\\
  \texttt{gold} & $\gold$ & nombre d’or\\
  \texttt{kro\{i\}\{j\}} & $\kro{i}{j}$ & symbole de Kronecker\\
  \multicolumn{3}{c}{\emph{Ensembles}}\\
  \midrule
  \multicolumn{3}{c}{\strong{Algèbre}}\\
  \multicolumn{3}{c}{\emph{Algèbre générale}}\\
  \multicolumn{3}{c}{\emph{Algèbre linéaire}}\\
  \midrule
  \multicolumn{3}{c}{\strong{Analyse}}\\
  \multicolumn{3}{c}{\emph{Calcul différentiel}}\\
  \multicolumn{3}{c}{\emph{Topologie}}\\
  \midrule
  \multicolumn{3}{c}{\strong{Probabilités}}\\
  \midrule
  \multicolumn{3}{c}{\strong{Arithmétique}}\\
  \midrule
  \texttt{Zmod\{n\}} &  $\Zmod{n}$ & classe d’équivalence modulo $n$ \\
  \texttt{Primes} & $\Primes$ & ensemble des nombres premiers \\
  \texttt{divides} & $d\divides a$ & relation de divisibilité \\
  \texttt{congr\{a\}\{b\}\{n\}} & $\congr{a}{b}{n}$ & $a$ congru à $b$ modulo $n$\\
  \texttt{pgcd\{a\}\{b\}} & $\pgcd{a}{b}$ & PGCD \\
  \texttt{ppcm\{a\}\{b\}} & $\ppcm{a}{b}$ & PPCM \\
  \bottomrule
\end{tabular}

\end{document}
