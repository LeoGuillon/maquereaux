\documentclass{article}

\usepackage{maquereaux}

\usepackage{booktabs}

% \usepackage{sectsty}
% \allsectionsfont{\sffamily}

\usepackage{fontspec}
\setmainfont{IBM Plex Serif}
\setsansfont{IBM Plex Sans}
\setmonofont{IBM Plex Mono}

\usepackage{unicode-math}
\setmathfont{IBM Plex Math}

\usepackage[margin=0.5in]{geometry}

% ────────────────────────────────────────────────────────────────────────────────
% (METADATA)

\title{Commandes du package \texttt{maquereaux.sty}}
\author{Léo Guillon}
\date{}

\begin{document}

\maketitle

\section{Logique de conception des commandes}

De manière générale, on essaie de suivre les conventions mathématiques : une commande pour un ensemble est en majuscule, tandis qu’une fonction ou un opérateur sera en minuscule.

Par ailleurs, pour être le plus consistant possible avec les autres commandes déjà existantes en LaTeX, les commandes doivent être nommées en anglais.

\section{Listes des commandes}
\centering
\begin{tabular}{lcl}
  \toprule
  Commande & Affichage & Signification\\
  \midrule
  \multicolumn{3}{l}{\strong{Généralités}}\\
  \multicolumn{3}{l}{\emph{Constantes mathématiques}}\\
  \texttt{e} & $\e$ & constante exponentielle\\
  \texttt{i} & $\i$ & nombre $\i$\\
  \texttt{gold} & $\gold$ & nombre d’or\\
  \texttt{kro\{i\}\{j\}} & $\kro{i}{j}$ & symbole de Kronecker\\
  \texttt{longto} & $\longto$ & longue flèche \\
  \texttt{function\{f\}\{A\}\{B\}\{x\}\{f(x)\}} & $\function{f}{A}{B}{x}{f(x)}$ & définition de fonction \\
  \multicolumn{3}{l}{\emph{Ensembles}}\\
  \texttt{N} & $\N$ & ensemble des entiers naturels\\
  \texttt{Z} & $\Z$ & ensemble des entiers relatifs\\
  \texttt{Q} & $\Q$ & ensemble des nombres rationnels\\
  \texttt{R} & $\R$ & ensemble des nombres réels\\
  \texttt{C} & $\C$ & ensemble des nombres complexes\\
  \texttt{H} & $\H$ & ensemble des quaternions\\
  \texttt{card\{E\}} & $\card{E}$ & cardinal de l’ensemble $E$\\
  \texttt{parts\{E\}} & $\parts{E}$ & ensemble des parties de l’ensemble $E$\\
  \texttt{comp\{E\}} & $\comp{E}$ & complémentaire de l’ensemble $E$\\
  \midrule
  \multicolumn{3}{l}{\strong{Algèbre}}\\
  \multicolumn{3}{l}{\emph{Algèbre générale}}\\
  \texttt{Sym} & $\Sym$ & groupe symétrique\\
  \texttt{Alt} & $\Alt$ & groupe alterné\\
  \multicolumn{3}{l}{\emph{Algèbre linéaire}}\\
  \midrule
  \multicolumn{3}{l}{\strong{Analyse}}\\
  \multicolumn{3}{l}{\emph{Calcul différentiel}}\\
  \multicolumn{3}{l}{\emph{Topologie}}\\
  \midrule
  \multicolumn{3}{l}{\strong{Probabilités}}\\
  \midrule
  \multicolumn{3}{l}{\strong{Arithmétique}}\\
  \texttt{Zmod\{n\}} &  $\Zmod{n}$ & classe d’équivalence modulo $n$ \\
  \texttt{Primes} & $\Primes$ & ensemble des nombres premiers \\
  \texttt{divides} & $d\divides a$ & relation de divisibilité \\
  \texttt{congr\{a\}\{b\}\{n\}} & $\congr{a}{b}{n}$ & $a$ congru à $b$ modulo $n$\\
  \texttt{pgcd\{a\}\{b\}} & $\pgcd{a}{b}$ & PGCD \\
  \texttt{ppcm\{a\}\{b\}} & $\ppcm{a}{b}$ & PPCM \\
  \bottomrule
\end{tabular}

\end{document}
