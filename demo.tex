\documentclass[print]{atomathematyk}

\usepackage{longtable}

% ────────────────────────────────────────────────────────────────────────────────
% (METADATA)

\title{Commandes du package \texttt{maquereaux.sty}}
\author{Léo Guillon}
\date{}

\begin{document}

\maketitle

\section{Logique de conception des commandes}

De manière générale, on essaie de suivre les conventions mathématiques : une commande pour un ensemble est en majuscule, tandis qu’une fonction ou un opérateur sera en minuscule.

Par ailleurs, pour être le plus consistant possible avec les autres commandes déjà existantes en \LaTeX, les commandes doivent être nommées en anglais, dans la mesure du possible et de la convenance.

\section{Listes des commandes}
\centering
\begin{longtable}{lcl}
  \toprule
  Commande & Affichage & Signification\\
  \midrule
  \multicolumn{3}{l}{\strong{Mise en forme mathématique}}\\
  \texttt{mathbi\{X\}} & \(\mathbi{X}\) & lettre en gras et italique\\
  \texttt{mathens\{N\}} & \(\mathens{N}\) & notation des ensembles usuels\\
  \midrule
  \multicolumn{3}{l}{\strong{Généralités}}\\
  \multicolumn{3}{l}{\emph{Constantes mathématiques}}\\
  \texttt{e} & \(\e\) & constante exponentielle\\
  \texttt{i} & \(\i\) & nombre \(\i\)\\
  \texttt{gold} & \(\gold\) & nombre d’or\\
  \multicolumn{3}{l}{\emph{Opérateurs génériques}}\\
  \texttt{kro\{i\}\{j\}} & \(\kro{i}{j}\) & symbole de Kronecker\\
  \texttt{ind} & \(\ind\) & fonction indicatrice\\
  \midrule
  \multicolumn{3}{l}{\strong{Théorie des ensembles}}\\
  \texttt{longto} & \(\longto\) & longue flèche \\
  \texttt{function\{f\}\{A\}\{B\}\{x\}\{f(x)\}} & \(\function{f}{A}{B}{x}{f(x)}\) & définition de fonction \\
  \multicolumn{3}{l}{\emph{Ensembles usuels}}\\
  \texttt{N} & \(\N\) & entiers naturels\\
  \texttt{Z} & \(\Z\) & entiers relatifs\\
  \texttt{Q} & \(\Q\) & nombres rationnels\\
  \texttt{R} & \(\R\) & nombres réels\\
  \texttt{C} & \(\C\) & nombres complexes\\
  \texttt{H} & \(\H\) & quaternions\\
  \multicolumn{3}{l}{\emph{Opérateurs ensemblistes}}\\
  \texttt{card\{E\}} & \(\card{E}\) & cardinal de l’ensemble \(E\)\\
  \texttt{parts\{E\}} & \(\parts{E}\) & ensemble des parties de l’ensemble \(E\)\\
  \texttt{comp\{E\}} & \(\comp{E}\) & complémentaire de l’ensemble \(E\)\\
  \midrule
  \multicolumn{3}{l}{\strong{Algèbre}}\\
  \multicolumn{3}{l}{\emph{Algèbre générale}}\\
  \texttt{Sym} & \(\Sym\) & groupe symétrique\\
  \texttt{Alt} & \(\Alt\) & groupe alterné\\
  \texttt{iso} & \(\iso\) & relation d’isomorphisme\\
  \texttt{semiprod} & \(\semiprod\) & produit semi-direct de groupes\\
  \texttt{subgroup} & \(\subgroup\) & relation de sous-groupe\\
  \texttt{normal} & \(\normal\) & relation de sous-groupe normal\\
  \texttt{centre\{G\}} & \(\centre{G}\) & centre du groupe \(G\).\\
  \texttt{Hom} & \(\Hom\) & (homo)morphismes de groupes \\
  \texttt{Iso} & \(\Iso\) & isomorphismes de groupes \\
  \texttt{End} & \(\End\) & endomorphismes de groupes \\
  \texttt{Aut} & \(\Aut\) & automorphismes de groupes \\
  \multicolumn{3}{l}{\emph{Algèbre linéaire}}\\
  \texttt{Lin} & \(\Lin\) & ensemble d’applications linéaires\\
  \texttt{Quad} & \(\Quad\) & ensemble de formes quadratiques\\
  \texttt{dual\{E\}} & \(\dual{E}\) & dual de l’espace vectoriel \(E\)\\
  \texttt{M} & \(\M\) & ensemble de matrices\\
  \texttt{GL} & \(\GL\) & groupe linéaire\\
  \texttt{SL} & \(\SL\) & groupe spécial linéaire\\
  \texttt{Orth} & \(\Orth\) & groupe orthogonal\\
  \texttt{SO} & \(\SO\) & groupe spécial orthogonal\\
  % \texttt{t\{A\}} & \(\t{A}\) & transposée de la matrice \(A\)\\
  \texttt{tr} & \(\tr\) & trace\\
  \texttt{ker} & \(\ker\) & noyau\\
  \texttt{im} & \(\im\) & image\\
  \texttt{rg} & \(\rg\) & rang\\
  \texttt{codim} & \(\codim\) & codimension\\
  \texttt{com} & \(\com\) & comatrice\\
  \texttt{spectrum} & \(\spectrum\) & spectre\\
  \texttt{spradius} & \(\spradius\) & rayon spectral\\
  \midrule
  \multicolumn{3}{l}{\strong{Analyse}}\\
  \multicolumn{3}{l}{\emph{Calcul différentiel et intégral}}\\
  \texttt{d} & \(\d\) & opérateur différentiel élémentaire\\
  \texttt{diff\{f\}\{a\}} & \(\diff{f}{a}\) & différentielle de \(f\) en \(a\)\\
  \texttt{grad} & \(\grad\) & gradient\\
  % \texttt{div} & \(\div\) & divergence\\
  \texttt{rot} & \(\rot\) & rotationnel\\
  \texttt{lap} & \(\lap\) & laplacien\\
  \texttt{jac} & \(\jac\) & matrice jacobienne\\
  \texttt{detjac\{f\}\{a\}} & \(\detjac{f}{a}\) & déterminant jacobien de \(f\) en \(a\)\\
  \texttt{hess\{f\}} & \(\hess{f}\) & matrice hessienne de \(f\) en \(a\)\\
  \multicolumn{3}{l}{\emph{Topologie}}\\
  \texttt{interior\{A\}} & \(\interior{A}\) & intérieur de \(A\)\\
  \texttt{adh\{A\}} & \(\adh{A}\) & adhérence de \(A\)\\
  \texttt{front\{A\}} & \(\front{A}\) & frontière de \(A\)\\
  \texttt{abs\{x\}} & \(\abs{x}\) & valeur absolue (ou module) de \(x\)\\
  \texttt{norme\{x\}} & \(\norme{x}\) & norme de \(x\)\\
  \midrule
  \multicolumn{3}{l}{\strong{Probabilités}}\\
  \multicolumn{3}{l}{\emph{Opérateurs usuels}}\\
  \texttt{Prob\{A\}} & \(\Prob{A}\) & probabilité d’un évènement \(A\)\\
  \texttt{Esp\{X\}} & \(\Esp{X}\) & espérance d’une variable aléatoire \(X\)\\
  \texttt{Var\{X\}} & \(\Var{X}\) & variance d’une variable aléatoire \(X\)\\
  \texttt{sd\{X\}} & \(\sd{X}\) & écart-type d’une variable aléatoire \(X\)\\
  \multicolumn{3}{l}{\emph{Lois discrètes usuelles}}\\
  \texttt{Bernoulli\{p\}} & \(\Bernoulli{p}\) & loi de Bernoulli de paramètre \(p\)\\
  \texttt{Binom\{n\}\{p\}} & \(\Binom{n}{p}\) & loi binomiale de paramètres \((n,p)\)\\
  \texttt{Poisson\{\backslash lambda\}} & \(\Poisson{\lambda}\) & loi de Poisson de paramètre \(\lambda\)\\
  \texttt{Geom\{p\}} & \(\Geom{p}\) & loi géométrique de paramètre \(p\)\\
  \texttt{Hyper\{N\}\{n\}\{k\}} & \(\Hyper{N}{n}{k}\) & loi hypergéométrique de paramètres \((N,n,k)\)\\
  \multicolumn{3}{l}{\emph{Lois continues usuelles}}\\
  \texttt{Exp\{\backslash lambda\}} & \(\Exp{\lambda}\) & loi exponentielle de paramètre \(\lambda\)\\
  \texttt{Normale\{\backslash mu\}\{\backslash sigma 2\}} & \(\Normale{\mu}{\sigma^2}\) & loi normale de paramètres \((\mu, \sigma)\)\\
  \texttt{chid\{n\}} & \(\chi2{n}\) & loi du \(\chi^2\) à \(n\) degrés de liberté\\
  \midrule
  \multicolumn{3}{l}{\strong{Arithmétique}}\\
  \texttt{Zmod\{n\}} &  \(\Zmod{n}\) & classe d’équivalence modulo \(n\) \\
  \texttt{Primes} & \(\Primes\) & ensemble des nombres premiers \\
  \texttt{divides} & \(\divides\) & relation de divisibilité \\
  \texttt{congr\{a\}\{b\}\{n\}} & \(\congr{a}{b}{n}\) & \(a\) congru à \(b\) modulo \(n\)\\
  \texttt{pgcd\{a\}\{b\}} & \(\pgcd{a}{b}\) & PGCD \(a\) et \(b\)\\
  \texttt{ppcm\{a\}\{b\}} & \(\ppcm{a}{b}\) & PPCM \(a\) et \(b\)\\
  \bottomrule
\end{longtable}

\end{document}
